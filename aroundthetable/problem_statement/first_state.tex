\documentclass[12pt,varwidth=48cm]{standalone}

\RequirePackage{times}
\RequirePackage{amsmath}
\RequirePackage{amssymb}
\RequirePackage[T1]{fontenc}

\usepackage[dvipsnames]{xcolor}
\usepackage{tikz}
\usetikzlibrary{backgrounds}
\usetikzlibrary{fadings}
\usepackage{pgfplots}
\pgfplotsset{compat=1.16}

\definecolor{red}{HTML}{972e21}
\definecolor{yellow}{HTML}{ebb83f}
\definecolor{blue}{HTML}{7999db}

\definecolor{green}{HTML}{5fd94e}
\definecolor{cyan}{HTML}{58c9c7}
\definecolor{magenta}{HTML}{d252d3}
\definecolor{lime}{HTML}{96e049}
\definecolor{orange}{HTML}{e78543}
\definecolor{violet}{HTML}{5d5bc4}
\definecolor{purple}{HTML}{9058cb}

\tikzset{>=stealth}

\tikzfading[
	name=lrballfading,
	left color=transparent!100,
	right color=transparent!0,
	middle color=transparent!33,
]

\tikzfading[
name=rlballfading,
left color=transparent!0,
right color=transparent!100,
middle color=transparent!33,
]

\newcounter{state}

\begin{document}
	\Large%
	\newcommand{\state}[3]{%
		\stepcounter{state}%
		\begin{tikzpicture}[line width=0.04cm]
			\newcommand{\drawball}[1]{
				\fill[lime!50!GreenYellow] (##1) circle (5pt);
				\begin{scope}[rotate around={rnd*360:(##1)}]
					\clip (##1) circle (5pt);
					\draw[white] ([shift={(8.5pt,8.5pt)}]##1) circle (10pt);
					\draw[white] ([shift={(-8.5pt,-8.5pt)}]##1) circle (10pt);
				\end{scope}
				\draw (##1) circle (5pt);
			}
			\clip (7,2.25) rectangle (-6,-2.75);
			\fill[green!70!black] (2,1.115) rectangle (-2,-1.115);
			\draw[white,line width=0.15cm] (2,1.115) rectangle (-2,-1.115);
			\draw[white,line width=0.06cm] (2,0) rectangle (-2,-0);
			\draw[black] (2,1.115) rectangle (-2,-1.115);
			\draw[dashed] (0,1.115) -- (0,-1.115);
			\def\xpos{-2.75}
			\foreach \name in {#2} {
				\colorlet{namecolor}{blue}
				\ifnum\pdfstrcmp{\name}{C}<0\colorlet{namecolor}{red!75}\fi
				\node[draw,circle,inner sep=0, outer sep=0,minimum width=1cm,fill=namecolor] at (\xpos,0) {\name};
				\pgfmathsetmacro\tmp{\xpos-1.25}
				\global\let\xpos\tmp
			}
			\ifnum\pdfstrcmp{#1}{<-}=0
				\begin{scope}[on background layer]
					\draw[->,rounded corners=0.75cm,line width=0.04cm] (2.75,0) --++ (0,1.5) -- (\xpos-0.25,1.5) -- (\xpos-0.25,0) --++ (0.8,0);
				\end{scope}
				\draw[black,path fading=rlballfading] (1.75,0.2) to[out=190,in=50] (1.25,-0.1) to[bend right] (-0.3,0.2);
				\drawball{-0.3,0.2}
			\fi
			\def\xpos{2.75}
			\foreach \name in {#3} {
				\colorlet{namecolor}{blue}
				\ifnum\pdfstrcmp{\name}{C}<0\colorlet{namecolor}{red!75}\fi
				\node[draw,circle,inner sep=0, outer sep=0,minimum width=1cm,fill=namecolor] at (\xpos,0) {\name};
				\pgfmathsetmacro\tmp{\xpos+1.25}
				\global\let\xpos\tmp
			}
			\ifnum\pdfstrcmp{#1}{->}=0
				\begin{scope}[on background layer]
					\draw[->,rounded corners=0.75cm,line width=0.04cm] (-2.75,0) --++ (0,-1.5) -- (\xpos+0.25,-1.5) -- (\xpos+0.25,0) --++ (-0.8,0);
				\end{scope}			
				\draw[black,path fading=lrballfading] (-1.75,0.2) to[out=-10,in=130] (-1.25,-0.1) to[bend left] (0.3,0.2);
				\drawball{0.3,0.2}
			\fi
			\foreach \name in {#2} {
				\global\let\leftname\name
				\breakforeach
			}
			\foreach \name in {#3} {
				\global\let\rightname\name
				\breakforeach
			}
			\node at (0,-2) {hit \thestate: \leftname{} and \rightname{} are facing each other};
		\end{tikzpicture}%
	}
	\state{->}{A,B}{D,E}%AD
%	\state{<-}{B}{D,E,A}%BD
%	\state{->}{B,D}{E,A}%BE
%	
%	\state{<-}{D}{E,A,B}%DE
%	\state{->}{D,E}{A,B}%ad
%	\state{<-}{E}{A,B,D}%EA
%	
%	\state{->}{E,A}{B,D}%be
%	\state{<-}{A}{B,D,E}%AB
%	\state{->}{A,B}{D,E}%ad
%	\begin{tikzpicture}[overlay]
%		\fill[white,opacity=0.7] (0,0) rectangle (-13,4.5);
%	\end{tikzpicture}%
\end{document}