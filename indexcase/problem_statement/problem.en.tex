\problemname{Index Case}

The epidemiologist W. Andy wants to find the index case of an ongoing crisis.
To do this, he modelled the city of the outbreak and its $n$ residents with a \emph{cellular automaton}.
The city is represented by $n$ cells numbered from $1$ to $n$ and each cell has two neighbouring cells, one to its left and one to its right.
The left neighbour of cell $i$ is cell $i-1$ and the right neighbour is cell $i+1$. Additionally, the left neighbour of cell $1$ is cell $n$ and the right neighbour of cell $n$ is cell $1$.
Thus, the city and the corresponding automaton form a simple cycle.

Each cell contains an integer between $1$ and $m$ which represents how likely it is that this person is infected.
Since the virus can only be transmitted by personal contact, the value in the $i$th cell on day $d$ only depends on the values of its neighbours and itself on the previous day.
If we denote this value by $s_{d}[i]$, then the outbreak can be simulated by a function $f$ using the formula:
\[s_{d}[i]=f\big(s_{d-1}[i-1],s_{d-1}[i],s_{d-1}[i+1]\big).\]
Note that as the city is cyclic both $i+1$ and $i-1$ are calculated modulo $n$.

Andy wants to find the index case, so he first has to find $s_0$, the state of the city on day zero.
This poses a problem, however, as it is not known on which day the crisis started.
Right now, Andy believes that he accomplished the task and found the state $s_0$, but you are not convinced.
Therefore, you want to check if there may be a state previous to the initial state proposed by Andy, i.e.\ whether there exists any state $s_{-1}$ that gets transformed into $s_0$ by applying $f$.

\section*{Input}
The input consists of:
\begin{itemize}
	\item One line with two integers $n$ and $m$ $(3\leq n \leq 200, 2\leq m \leq 10)$, the number of cells and the number of states.
  \item $m^3$ lines describing the values $f(x,y,z)$ ($1 \le f(x,y,z) \le m$ for each $1 \le x,y,z \le m$) of the function $f$ modelling the automaton.
    The values are given in lexicographic order of the arguments:
    The first value is $f(1,1,1)$, the next is $f(1,1,2)$, and so on until $f(1,1,m)$, followed by $f(1,2,1)$ and so forth. The last value is $f(m,m,m)$.
  \item One line with $n$ integers $s_0[1],\dots,s_0[n]$ ($1 \le s_0[i] \le m$ for each $i$), the initial state that has been proposed by Andy.
\end{itemize}

\section*{Output}
Output \texttt{yes} if there exists at least one possible previous state and \texttt{no} otherwise.
