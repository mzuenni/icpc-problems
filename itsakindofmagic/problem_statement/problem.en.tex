\problemname{}

\illustration{0.30}{parker.jpeg}{
	The Parker Square.
	\textcopyright{} \href{https://www.bradyharanblog.com/the-parker-square}{Brady Haran}, used with permission
}

%Count how many Multiplicative magic $3x3$ squares exists.
Everyone knows that a $3\times3$ magic square must meet two criteria:
\begin{enumerate}
	\item All nine numbers must be positive and distinct.
	\item The sums along all rows, columns, and diagonals are equal.
\end{enumerate}
Everyone, except  maybe Matt Parker\footnote{Recreational mathematician, author, comedian, YouTube personality, and science communicator.}.
He wants to create a magic square of squares, that is, a magic square that also meets a third criterion:
\begin{enumerate}[resume]
	\item Each number is a square of a positive integer.
\end{enumerate}
His ``result'' can be seen in the picture in the corner.
As you may notice, his square is not that magic\dots{}
Not only do most of the values appear twice, it also has a diagonal with the wrong sum.
To be honest, apart from containing non-square values, there is not much that could make this square worse.
Well, at least he tried!

But that is all in the past.
After finding the \emph{Parker Square}, he decided to completely ignore property $3$ from now on and to instead give property $2$ a new twist.
He now considers multiplicative magic squares, which are exactly like normal magic squares except that the \emph{products} along all rows, columns, and diagonals have to be equal, instead of the sums.
Who knows, Matt might even manage to find a proper multiplicative magic square in the future!
% Please don't find one before him.

With this definition at hand, Matt wrote some terrible Python code -- his words, not ours -- to count the number of multiplicative magic $3\times 3$ squares with product at most $n$.
As you may have guessed by now, his code is way too slow.
Therefore, we task you to do the same, just more efficiently.
Given an integer $n$, count the number of multiplicative magic $3\times 3$ squares with product at most $n$.

\begin{Input}
	The input consists of:
	\begin{itemize}
		\item One line with an integer $t$ ($1 \leq t \leq 10^5$), the number of test cases.
    \item $t$ lines, each with an integer $n$ ($1 \leq n \leq 10^{18}$), the maximum product.
	\end{itemize}
\end{Input}

\begin{Output}
	For each test case, output the number of multiplicative magic squares with product at most $n$.
\end{Output}
