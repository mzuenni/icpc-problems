\problemname{}

\illustration{0.3}{image}{Image by DGJ, \href{https://pixabay.com/users/gdj-1086657/}{pixabay.com}\vspace{-0.5cm}}%
%TODO
Chinese New Year is in four days and Lina is already excited about all the festivities.
She looks forward to eating dumplings on New Year's Eve and setting off fireworks to greet the new year.
And of course, the money she gets in small red bags from her relatives is always a plus.

Her whole family is already busy with the preparations such as cleaning and decorating the house.
Each year, they put up Chinese Spring Couplets next to their front door which will bring luck and prosperity next year.
For the first time, it is Lina's job to write the couplets, and she takes it very seriously.
However, Lina did not yet start her preparations and now she has to hurry up to finish before New Year's Eve.
Instead of following the very strict rules in which order the strokes of a glyph have to be written, Lina decides to speed things up by bending the rules a little.
She just wants to draw all the glyphs with as few strokes as possible without her parents noticing it.

% Right when you started with this years wintercontest you had a sudden realization: there are only $4$ days left until Chinese new year.
% This wouldn't be a big issue if you had not made a bet last year that you could easily learn to write Chinese in one year.
% Now you only have $4$ days left and and didn't even start yet.
% With time pressuring you you had a great idea: instead of properly learning to write Chinese you just want to learn to reproduce glyphs.
% To make this easier you decide to to decompose all the glyphs into the minimal number of strokes.

\begin{figure}[h]
	\centering
	\includegraphics[width=0.5\textwidth]{sample}
	\caption{Visualization of the second sample.
	Each colour represents one stroke.}
\end{figure}

But first, Lina needs to find out how to do this.
More formally: given a drawing of a glyph as a set of segments, Lina wants to decompose it into some paths that can each be drawn without lifting the pen.
Note that each segment should be drawn exactly once in the end so that no one will notice Lina's little trick.

\begin{Input}
	The input consists of:
	\begin{itemize}
		\item One line with two integers $n$ and $m$ ($4\leq n \leq10^5$, $5\leq m\leq3\cdot10^5$), the number of coordinates and the number of line segments in the drawing.
		\item $n$ lines, each with two integers $x$ and $y$ ($0\leq x,y\leq10^9$), the $i$th line describes the position of the $i$th coordinate.
		All coordinates are distinct.
		\item $m$ lines, each with two integers $a$ and $b$ ($1\leq a,b\leq n$, $a\neq b$), describing that the $a$th and $b$th coordinate are connected by a line segment.
		Two coordinates are connected by at most one line segment.
	\end{itemize}
	It is guaranteed that two lines segments only intersect in a coordinate they both start or end at.
	Further, it is guaranteed that each coordinate is incident to at least one line segment.
\end{Input}

\begin{Output}
	First, output a single integer, the minimum number of strokes needed to recreate the drawing.
	For each stroke first output the number of coordinates that the stroke consists of, followed by the actual coordinates along the stroke.
	If there are multiple optimal solutions, you may output any one of them.
\end{Output}

\nextsample{}

\par\vspace{0.5cm}
\begin{tabular}{l}
	\textbf{Sample Input 2}\\
\end{tabular}

We have a truly marvellous second sample for this Problem, which this page is too narrow to contain.
You can find the sample in the DOMjudge.

\newsavebox\removesample
\sbox\removesample{\remainingsamples{}}


